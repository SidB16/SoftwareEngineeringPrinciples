% !TeX root = ./projectStudy.tex
\documentclass{article}
\usepackage{graphicx}
\graphicspath{ {/home/sid16/Desktop/3342/project/images/} }
\begin{document}
  \tableofcontents
  \newpage
  
  \section{The Requirements Document}
  \begin{enumerate}
      \item Phases of the Life Cycle of Program Development:\begin{itemize}
          \item system analysis- feasibility study of system we want to construct.
          \item \textbf{requirements document- states clearly what the functions and constraints of the system are.(Written in Natural language)}
          \item technical specification- contains structured formalization of the previous document using some modelling technique.
          \item design- develops the previous phase by taking and justfying the decisions which impelment the previous specification, and 
          also defines the architecture of the future systems.
          \item implementation- contains the translation of the outcome of the previous phase into hardware and software components.
          \item tests- consists of the experimental verifications of the final system.
          \item maintenance- contains the system upgrading.
      \end{itemize}
      \item Difficulties with the Requirements Doc?\begin{itemize}
        \item What is difficult for the reader of the requirement document is to make distinctions between which part of text
        is devoted to \textbf{explanations} and which is devoted to genuine \textbf{requirements}.
        \item Explanations are needed initially for the reader to
        understand the future system. But when the reader is more acquainted
        with the purpose of the system, explanations are less important.
      \end{itemize}
      \item In mathematical texts, requirements are Defintions and Theorems. \begin{itemize}
        \item Such items are usually easily recognizable because they are labeled by their function(defition, lemma, theorem).
        \item Also, numbered in systematic fashion.
        \item Usually differs in font which differs from that used elsewhere in the book
        \item \includegraphics[width=\textwidth]{Cantor-Bernstein.png}
        \item \begin{verbatim}
          we can clearly see the “requirement” as in-dicated on the first line: 
          the theorem number, the theorem name, and the theorem statement (written initalic). 
          Next are the associated “explanations”: historical comments and proof.
        \end{verbatim}
      \end{itemize}
      \item the idea of stucturing the reqequirements doc is to have our requirement document organized around two texts embedded in each other: the explanatory text and the reference text.\begin{itemize}
        \item These two texts should be immediately separable, so that it is possible to summarize the reference text independently.
        \item reference text takes the form of labeled and numbered short statements written using natural language, , which must be very easy to read independently from the explanatory text.
        \item We shall use a special font for the reference text. 
        \item These fragments must be self contained without the explanations. 
        \item They together form the requirements. 
        \item The explanations are just there to give some comments which could help a first reader. But after an initial period, the reference text is the only one that counts.
      \end{itemize}
      \item Labeling and Numbering the requiremnts FUN: for functional requirements; ENV:for environment requiremnts, is important for traceability.\begin{itemize}
        \item So that it will be easy to recognize how each requirement has indeed been taken into account during the construction of our system and in its final operational version.
      \end{itemize}
      \item 
  \end{enumerate}
\end{document}